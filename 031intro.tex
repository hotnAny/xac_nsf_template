\section{Introduction}



\subsection{Intellectual Merits}


\subsection{Broader Impacts}


\subsection{Educational integration}

% Human computer interaction (HCI) is uniquely valuable across a broad range of instructional settings in that it connects to a breadth of STEM subdisciplines at various levels, providing a compelling and engaging front-end within which to explore educational objectives.  
% In developing and advancing our research goals, we have and will continue to make extensive use of high school, undergraduate, and graduate student researchers to investigate and explore specific academic subproblems under the proposed tasks.  These students will participate both through extracurricular and outreach programs as well as through design-based courses integrated into the degree curriculum.  Computational design coursework developed and taught by PI Mehta and HCI coursework developed and taught by PI Chen provide additional opportunities to engage students through curriculum design and teaching assistantships.  We will further collaborate with design capstone courses in the Mechanical and Aerospace Engineering (MAE) department at UCLA (see attached letter of collaboration) to expand the target scope of our contributions.  

% The outcomes of this proposed research also provide further avenues for educational development.  The design tools resulting from our work enable the widespread accessibilty of hands-on engineering instruction by lowering barriers to creation.  Our tools can be distributed to participants in massive open online courses (MOOCs), made freely available for use in grade school classrooms around the world (and especially in low-resource and underprivileged demographics), and form the focus of dedicated hackathons and short courses for targeted impact especially aimed at engaging women and minorities.

