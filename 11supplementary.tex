\noindent{\Large \bf Supplementary Documents - 3. Data Management Plan}\\

The PI is committed to NSF's AAG policy of prompt publication of sponsored research data and has a history of actively sharing and communicating the results of his research with the scientific community via conferences and other synergistic activities. \\

\textbf{Expected data to be managed:}
The research conducted by this infrastructure will generate two classes of data: experimental measurements and results, and software repositories.  

The experimental data will include primary data generated by the infrastructure over the course of an experiment, along with the associated metadata describing the experimental setup, experimental methods and procedure, mathematical models, and data analysis techniques.  

The software repositories will include code written to prepare experiments, execute tasks and gather data, and analyze and present results.  They may also include user-facing applications and systems.   \\

\textbf{Period of retention:} 
Data will be retained for at least three years after the conclusion of the award or three years after public release, whichever is later.  Data related to a student's work will be retained for a minimum of three years after the degree is awarded.   \\

\textbf{Data storage:}
These data will be stored in raw form on the experimenter's computing platform and backed up to a secure cloud storage.  A copy of all prepared data will additionally be stored on a lab-managed server in the office of the PI.  In the event of a personnel change, the PI, or in their absence the School of Engineering and Applied Sciences at UCLA, will be responsible for maintaining custodianship of all data. \\

\textbf{Sharing and access:}
The experimental data will be promptly prepared and published in the form of conference and journal papers, theses, and other publication formats.  The prepared form of this data will be made available electronically to the public from the lab website as well as electronically or in print from the publisher subject to subscription or printing charges.  The raw data may also be made available upon request, and transmitted via email.

The software repositories may similarly be published as above when appropriate, and may furthermore be made available as open source repositories through the lab servers or via public open source hosting services.  Specific requests for the software may also be entertained, and transmitted via email.

\newpage


\noindent{\Large \bf Supplementary Documents - 1. List of Project Personnel and Partner Institutions}\\

\title\\

\begin{enumerate}
  \item Xiang `Anthony' Chen; University of California, Los Angeles; co-PI
\end{enumerate}

\newpage

\noindent{\Large \bf Supplementary Documents - 2. Collaboration Plan}\\

\title\\

\paragraph{Team Composition and Member Roles.}  % Members and Specific Roles

PI Chen will supervise a full-time graduate student researcher focusing primarily on thrust 1.
PI Mehta will supervise a full-time graduate student researcher focusing primarily on thrust 2, and serve as the lead PI coordinating all research tasks across the group.  These focus targets will be mainly for an unambiguous hierarchy of responsibility; nonetheless, all team members will be engaged in and contribute to all aspects of the proposed research.  

\paragraph{Team Synergy and Collaboration Mechanism.}

The proposed project will be conducted at UCLA, the common institution of the two coPIs.  The team is already accustomed to regular face-to-face meetings with regards to prior and ongoing workshop and paper collaborations.  Members of the PIs' groups regularly interact in both academic and non-academic contexts.

This collaboration will continue for the proposed project, and happen at two levels: individual researcher interactions and project-wide team interactions. At the individual level, researchers will collaborate on their respective tasks (as identified in the proposal), co-author joint papers, and co-develop tools.  To support project-wide collaboration, there will be weekly all-hands meetings dedicated to ensuring all personnel are aware and engaged in project developments.  There will also be yearly retreats to promote cohesiveness as a team.  Finally, we will use various forms of collaboration tools including online document sharing, code repositories, real-time and asynchronous chat, and other cloud services.

%\paragraph{Project timeline and management plan}

%The proposed project timeline is displayed in the Gantt chart in \fgref{gantt}.
%\fg{0.9}{gantt}{A Gantt chart showing the project timeline}

\paragraph{Budget for Collaboration.}

As the two coPIs are at the same institution and share lab and office space, no particular budget is necessary for collaboration.  

\newpage


\noindent{\Large \bf Supplementary Documents - 4. Broadening Participation in Computing (BPC)}\\

% Each Medium/Large project must, by the time of award, have in place an approved BPC plan. In this ongoing pilot phase, CISE will work with each PI team prior to making an award to ensure that plans are meaningful and include concrete metrics for success. CISE will also provide opportunities for PIs to share BPC experiences and innovations through program PI meetings. PIs of Medium/Large proposals are therefore strongly encouraged to consider this eventual requirement as they develop their proposals, and to include descriptions (of one to three pages) of their planned BPC activities under Supplementary Documents in their submissions. Feedback will be provided on such plans.


The HCI group at UCLA led by PI Chen currently has 10 undergraduate students (50\%), six female members (28\%), and one hispanic student.  The robotics group of PI Mehta has included 33 undergraduate researchers, of which 13 (39\%) were female and 3 were from URM/POC demographics.

In this proposed project, we plan to broaden participation in computing by the following specific activities:
\begin{itemize}
	\item Collaboration with Dr. Xiaochun Li from UCLA Department of Mechanical and Aerospace Engineering (MAE). Our design tools will be used in Dr. Li's undergraduate design class as a way to expand MAE students' skill beyond conventional CAD tools, thus enabling them to explore a wider range of design possibilities that involve novel interactive computing and a tighter integration with physical objects and environment.
	\item Leveraging the recently-established Maker Space in the UCLA School of Engineering, we plan to deploy our interactive tools in quarterly workshops to engage campus-wide students to brainstorm ideas, create digital designs and fabricate physical objects.
\end{itemize}
 

% \textbf{Undergraduate Education:}
The PIs continue to be active in \textbf{undergraduate projects}.
We plan to engage students in the proposed research through a series of senor capstone projects. 
Building on past curriculum development and undergraduate research opportunities, we will foster a tradition of undergraduate involvement in this research program by incorporating coursework and extracurricular activities on design tools, immersive interfaces, and personalized creation.

We plan to aggressively advertise the summer programs and capstone projects described above to \textbf{women and underrepresented groups} through our existing associations with programs such as the UCLA Center for Excellence in Engineering and Diversity (CEED), UCLA High School Tech Camp, and the High School Summer Research Program, which focus on attracting and retaining these demographics in STEM fields.

PI Mehta has actively engaged with the \textbf{global public}, including community members and K-12 students, organizing open-houses and demonstrations to all demographics of visitors, along with hosting summer interns from high schools across the state.  Further lab tours and visits will be organized with local middle and high schools to demonstrate the beneficial effects of STEM education on society and encourage participation.  We will continue to seek such opportunities to engage with their local communities through additional events such as hackathons with specific focus problems for \textbf{citizen scientists} and \textbf{makers} to participate in personalized creation.  %I will also continue to bring robotics to {\bf underserved communities}, building on my award winning contributions to the African Robotics Network Ultra Affordable Educational Robot Project \citeref{afron}.

